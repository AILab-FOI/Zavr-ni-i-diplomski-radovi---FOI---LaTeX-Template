\documentclass{foi}
\usepackage[utf8]{inputenc}
\usepackage{lipsum}

\vrstaRada{\zavrsni} % \diplomski
\title{Naslov završnog/diplomskog rada -- LaTeX Predložak}

\author{Marija Horvat}
\mentor{Renata Mekovec}
\godina{2017}
\mjesec{rujan}
\date{2017}
%\status{redoviti}
\indeks{35918/07–R}
\smjer{Informacijski sustavi} % (ili Poslovni sustavi, Ekonomika poduzetništva, Primjena informacijske tehnologije u poslovanju, Informacijsko i programsko inženjerstvo, Baze podataka i baze znanja, Organizacija poslovnih sustava, Informatika u obrazovanju)
\titulaProfesora{Doc. dr. sc.}

\sazetak{Opsega od 100 do 300 riječi. Sažetak upućuje na temu rada, ukratko se iznosi čime se rad bavi, teorijsko-metodološka polazišta, glavne teze i smjer rada te zaključci.}

\kljucneRijeci{riječ; riječ; ...riječ; Obuhvaća 7+/-2 ključna pojma koji su glavni predmet rasprave u radu.}

\begin{document}

\maketitle

\tableofcontents

\pagestyle{plain}
\chapter{Uvod}

Završni ili diplomski rad studentice/studenta je konačni rezultat uloženog napora u završetak studija. Obranom završnog ili diplomskog rada studentica/student stječe prava i obveze koje proizlaze iz završetka akademskog obrazovanja. S ciljem osiguranja potpore studentima pri pisanju završnog/diplomskog rada, izrađen je ovaj predložak oblikovanja samog rada.

Načelna napomena o strukturi rada jest da se nazivi i struktura poglavlja obavezno definiraju u dogovoru s mentorom. Sadržajna preporuka je da u uvodu treba opisati što je tema završnog/diplomskog rada, zašto je tema značajna te koja je motivacija studenta za odabir teme. 

\chapter{Metode i tehnike rada}

U ovom poglavlju treba opisati koje će metode i tehnike biti korištene pri razradi teme, kako su provedene istraživačke aktivnosti, koji su programski alati ili aplikacije korišteni.

\lipsum[1-2]

\chapter{Razrada teme}

Ovo je glavni dio rada u kojem treba razraditi temu, pojasniti istraživanja, prikazati rezultate i slično. Poželjno je na početku poglavlja dati kratki opis strukture poglavlja, kako bi čitatelj rada mogao lakše pratiti složenu cjelinu.

\section{Poglavlje druge razine }

\lipsum[1]

\subsection{Poglavlje treće razine}

\lipsum[2]

\subsubsection{Poglavlje četvrte razine}

\lipsum[3-4]

\chapter{Tehničke upute}

Tehničke upute u nastavku opisuju način tehničkog oblikovanja rada i navođenja literature.

\section{Upute za oblikovanje izgleda rada}

\begin{flushleft}\textbf{Stranice} se oblikuju korištenjem sljedećih parametara:\end{flushleft}

\begin{itemize}
    \item veličina i oblik papira je A4, okomito usmjerenje, margine 2,5 cm na svakoj strani;

    \item naslovna stranica rada se ne numerira;

    \item nakon naslovne stranice, sve sljedeće stranice do 1. Poglavlja se numeriraju rimskim brojevima, počevši od i;

    \item od 1. poglavlja nadalje, stranice se numeriraju arapskim brojevima;

    \item broj stranice treba pozicionirati desno 1,25 cm od dna stranice, font Arial 9.
\end{itemize}
\begin{flushleft}\textbf{Tekst} rada je potrebno oblikovati sukladno ovom predlošku, odnosno na sljedeći način:\end{flushleft}
\begin{itemize}
    \item u pisanju teksta koristite font Arial 11 pt, s proredom 1,5 te razmakom 0 pt prije i razmakom 6 pt poslije odlomka, pri čemu je prvi redak uvučen za 1,25 cm;

    \item u naslovima prve razine „3. Razrada teme“ koristite font Arial 18 pt, podebljano, prijelom stranice (svaki naslov prve razine treba biti na novoj stranici), s proredom 1,5 te razmakom 0 pt prije i razmakom 18 pt poslije odlomka;

    \item u naslovima druge razine „2.1. Naslov“ koristite font Arial 16 pt, podebljano, s proredom 1,5 te razmakom 18 pt prije i razmakom 12 pt poslije odlomka;

    \item u naslovima treće razine „2.1.1. Naslov“ koristite font Arial 14 pt, podebljano, s proredom 1,5 te razmakom 12 pt prije i razmakom 6 pt poslije odlomka;

    \item u naslovima četvrte razine „2.1.1.1. Naslov“ koristite font Arial 12 pt, podebljano, s proredom 1,5 te razmakom 6 pt prije i razmakom 6 pt poslije odlomka;

    \item ostalo značajno isticanje cjelina rada može biti istaknuto podebljanim i kurziv slovima, korištenjem fonta Arial 11 pt.
\end{itemize}


\begin{flushleft}\textbf{Slike} u radu je potrebno oblikovati na sljedeći način:
naziv slike navedite ispod slike uz numeraciju;\end{flushleft}

\begin{itemize}
    \item za nazive slika koristite iste postavke fonta kao i za tekst, ali stavite naziv slike u centrirani položaj;

    \item za oblikovanje same slike koristite font Arial 9 pt za tekst na slici;
ispred same slike umetnite jedan prazan redak (osim ako je slika pozicionirana na početku stranice);

    \item nakon naziva slike ostavite jedan redak prazan (osim ako je naziv slike zadnji redak na stranici);

    \item kod prijeloma stranice treba obratiti posebnu pozornost da naziv slike, izvor i sama slika moraju biti na istoj stranici; 

    \item slike je potrebno numerirati redom pojavljivanja u tekstu;

    \item ako je slika preuzeta iz drugog izvora, nakon navođenja naziva slike u zagradi navedite izvor, npr. (Autor, godina);

    \item dozvoljeno je napraviti vlastitu preradu slika, grafikona ili tablica na način da se zadrži isti smisao sadržaja, ali promijeni izgled. I u takvim se slučajevima obavezno u nazivu navodi referenca izvornog djela ovako: “(Prema: Klačmer Čalopa i Cingula, 2012)“;

    \item dozvoljeno je preuzeti samo jednu sliku, grafikon ili tablicu u izvornom obliku iz istog izvora. Za doslovno preuzimanje većeg dijela sadržaja potrebno je ishoditi dozvolu nositelja autorskih prava;

    \item primjer označavanja slike možete vidjeti u nastavku (slika \ref{fig:podjela}).
\end{itemize}

\begin{figure}[h!]
    \centering
    \includegraphics[width=0.9\textwidth]{slike/slika.png}
    \caption{Podjela investicijskih fondova (Izvor: \citeauthor{Aranda2009}, \citeyear{Aranda2009})}
    \label{fig:podjela}
\end{figure}

\begin{flushleft}\textbf{Tablice} rada je potrebno oblikovati sukladno ovim uputama:\end{flushleft}
\begin{itemize}
    \item naziv tablice navedite iznad slike;

    \item za nazive tablica koristite iste postavke fonta kao i za tekst, ali stavite naziv tablice u centrirani položaj;

    \item za oblikovanje same tablice koristite font Arial 9 pt za tekst u tablici;

    \item tablice numerirajte redom pojavljivanja u tekstu;

    \item prije naziva tablice umetnite jedan redak prazan (osim ako je naziv tablice prvi redak na stranici);

    \item nakon same tablice umetnite jedan prazan redak (osim ako je tablica pozicionirana na kraju stranice);

    \item kod prijeloma stranice treba obratiti posebnu pozornost da naziv tablice, izvor i sama tablica moraju biti na istoj stranici; 

    \item ako je tablica preuzeta iz drugog izvora, nakon navođenja naziva tablice potrebno je navesti izvor, na isti način kako je opisano kod slika;

    \item primjer označavanja tablice možete vidjeti u nastavku (tablica \ref{tab:objekti}).
\end{itemize}

\begin{table}[h!] 
    \centering
    \caption{Prikaz podataka o učestalosti pojavljivanja objekta}
    \begin{tabularx}{0.66\textwidth}{|X|X|X|X|}
        \hline
         \cellcolor{gray!25} & \cellcolor{gray!25} & \cellcolor{gray!25} & \cellcolor{gray!25} \\
        \hline
         &  &  &  \\
        \hline
         &  &  & \\
        \hline
    \end{tabularx}
    \\[10pt]
    \caption*{(Izvor: \citeauthor{caragliu2011smart}, \citeyear{caragliu2011smart})}
    \label{tab:objekti}
\end{table}

\begin{flushleft}\textbf{Programski kod}\end{flushleft}
\begin{itemize}
    \item za oblikovanje teksta koji je programski kôd koristite font Courier, veličine 10 pt, jednostruki prored, 6 pt iza odlomka, npr. HTML kôd dijela zaglavlja početne web stranice FOI weba:
\end{itemize}

\begin{lstlisting}[language=HTML]
<head>
  <meta http-equiv="Content-Type" content="text/html; charset=utf-8" />
  <link rel="shortcut icon" href="https://www.foi.unizg.hr/sites/default/files/favicon_0_1.ico" type="image/vnd.microsoft.icon" />
  <meta name="generator" content="Drupal 7 (http://drupal.org)" />
  <link rel="canonical" href="https://www.foi.unizg.hr/hr" />
  <link rel="shortlink" href="https://www.foi.unizg.hr/hr" />
  <!-- Set the viewport width to device width for mobile -->
  <meta name="viewport" content="width=device-width, initial-scale=1.0">
  <title>Dobro %*došli*) na FOI | FOI</title>...
</head>
\end{lstlisting}

\begin{flushleft}\textbf{Formule}\end{flushleft}
\begin{itemize}
    \item za unos formula koristite editor za formule u svom tekst procesoru.
\end{itemize}

\begin{flushleft}\textbf{Kratice}\end{flushleft}   
\begin{itemize}
    \item ako želite koristiti kratice pojmova u tekstu, kad prvi put spominjete pojam potrebno je navesti puni naziv, a kraticu navesti u zagradi (npr. Informacijske i komunikacijske tehnologije, kraće IKT). Nakon toga možete koristiti kratice u tekstu. Poželjno je u naslovima koristiti pune nazive.
\end{itemize}

\begin{flushleft}\textbf{Strano nazivlje}\end{flushleft}   
\begin{itemize}
    \item strano nazivlje se u tekstu navodi u zagradi, napisano \textit{kurzivom}, nakon hrvatskog izraza, npr. Analiza društvene mreže (eng. \textit{Social Network Analysis - SNA}).
\end{itemize}

\section{Navođenje literature}

Za navođenje literature u radu možete odabrati i koristiti jedan od sljedeća dva  ponuđena stila: \textbf{APA} ili \textbf{IEEE} stil. Važno je samo dosljedno primjenjivati odabrani stil u cijelom radu.

U popisu literature potrebno je navesti svu literaturu i samo literaturu koju ste koristili u tekstu.

Uz svaku preuzetu tvrdnju potrebno je navesti njezin izvor, tj. referencu. Reference se u tekstu navode tako da se uz citirani tekst navede izvor, sukladno načinu propisanom odabranim stilom i FOI preporukama za citiranje i referenciranje \cite{SchattenEtAl2016roadmap}.

\chapter{Zaključak}

Ovdje treba sažeto rezimirati najvažnije rezultate razrade teme rada. Potrebno je sažeto opisati što je predmet rada, koje su metode, tehnike, programski alati ili aplikacije korištene u razradi rada te koje su pretpostavke dokazane, a koje opovrgnute. Sadržajno, ono što se u uvodu rada najavljuje i kasnije je obuhvaćeno u samom radu, moralo bi biti opisano u zaključnom dijelu kroz rezultate rada. 

\lipsum[1-2]

\printbibliography[title=Popis literature]
\addcontentsline{toc}{chapter}{Popis literature}

\listoffigures
\addcontentsline{toc}{chapter}{Popis slika}
 
\listoftables
\addcontentsline{toc}{chapter}{Popis popis tablica}

\appendix
\renewcommand{\thechapter}{\arabic{chapter}}

\chapter{Prilog 1}

\chapter{Prilog 2}

\end{document}
